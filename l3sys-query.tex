\iffalse meta-comment

File l3sys-query.tex Copyright (C) 2024 The LaTeX Project

Permission is hereby granted, free of charge, to any person obtaining a copy
of this software and associated documentation files (the "Software"), to deal
in the Software without restriction, including without limitation the rights
to use, copy, modify, merge, publish, distribute, sublicense, and/or sell
copies of the Software, and to permit persons to whom the Software is
furnished to do so, subject to the following conditions:

The above copyright notice and this permission notice shall be included in all
copies or substantial portions of the Software.

THE SOFTWARE IS PROVIDED "AS IS", WITHOUT WARRANTY OF ANY KIND, EXPRESS OR
IMPLIED, INCLUDING BUT NOT LIMITED TO THE WARRANTIES OF MERCHANTABILITY,
FITNESS FOR A PARTICULAR PURPOSE AND NONINFRINGEMENT. IN NO EVENT SHALL THE
AUTHORS OR COPYRIGHT HOLDERS BE LIABLE FOR ANY CLAIM, DAMAGES OR OTHER
LIABILITY, WHETHER IN AN ACTION OF CONTRACT, TORT OR OTHERWISE, ARISING FROM,
OUT OF OR IN CONNECTION WITH THE SOFTWARE OR THE USE OR OTHER DEALINGS IN THE
SOFTWARE.

-----------------------------------------------------------------------

The development version of the bundle can be found at

   https://github.com/latex3/l3sys-query

for those people who are interested.

-----------------------------------------------------------------------

\fi

\documentclass{l3doc}

\begin{document}

\title{The \pkg{l3sys-query} script: System queries for LaTeX using Lua}

\author{%
  The \LaTeX{} Project\thanks{%
    Email: \href{mailto:latex-team@latex-project.org}
      {latex-team@latex-project.org}%
  }%
}

\date{Released 2024-03-03}

\maketitle
\tableofcontents

\begin{documentation}

\section{Introduction}

\TeX{} engines provide only very limited access to information about the system
they are used on: using primitives, one can for example get the size of a single
file, but not a list of files in a given location. For most documents, this is
not an issue as they are self-contained. However, for cases where
\enquote{dynamic} construction of parts of a document is needed based on file
lists or other system-dependent data, methods to obtain this from (restricted)
shell escape are desirable.

Security considerations mean that directly querying the system shell is
problematic for general use. Instead, \emph{restricted} shell escape may be used
to get many details, provided a suitable tool is available to provide the
information in a platform-neutral and security-conscious way. The Java program
\texttt{texosquery}, written by Nicola Talbot, has been available for a number
of years to provide this facility. As well as file system insight,
\texttt{texosquery} also provides for example locale data and other system
information. However, the requirement for Java means that the script is not
automatically usable when a \TeX{} system is installed.

The \LaTeX{} team have therefore provided a Lua-based script,
\texttt{l3sys-query}, which conforms to the security requirements of \TeX{} Live
using Lua to obtain the system information. This means that it can be used
\enquote{out of the box} across platforms. The facilities provided by
\texttt{l3sys-query} are more limited than \texttt{texosquery}, partly as some
information is available in modern \TeX{} systems using primitives, and partly
as the aim of \texttt{l3sys-query} is to provide information where there are
defined use cases. Requests for additional data interfaces are welcome.

\section{The command line interface}

The command line interface to 
\begin{center}
  \ttfamily
  l3sys-query \meta{cmd} [\meta{option(s)}] [\meta{args}]
\end{center}
where \texttt{\meta{cmd}} can be one of the following:
\begin{itemize}[noitemsep]\ttfamily
  \item ls
  \item ls \meta{args}
  \item pwd
\end{itemize}
The \meta{cmd} are described below. The result of the \meta{cmd} will be printed
to the terminal in an interactive run; in normal usage, this will be piped to
the calling \TeX{} process. Results containing path separators \emph{always}
use~|/|, irrespective of the platform in use.

As well as these targets, the script recognizes the options
\begin{itemize}
  \item |--all| Include \enquote{dot} entries in directory listings
  \item |--exclude| Specification for directory entries to exclude
  \item |--ignore-case| Ignores case when sorting directory listings
  \item |--pattern| (|-p|) Treat the \meta{args} as Lua patterns rather
    than converting from wildcards
  \item |--recursive| (|-r|) Enables recursive searching during directory
    listings
  \item |--reverse-sort| Causes sorting to go from highest to lowest rather
    than lowest to highest
  \item |--sort| Sets the method used to sort entries returned by |ls|
  \item |--type| Selects the type of entry returned by |ls|
\end{itemize}
The action of these options on the appropriate \meta{cmd(s)} is detailed below.

\subsection{\texttt{ls [\meta{args}]}}

Lists the contents of one or more directories, in a manner somewhat reminiscent
of the Unix command |ls| or the Windows command |dir|. The exact nature of the
output will depend on the \meta{args}, if given, along with the prevailing
options.

When no \meta{args} are given, all entries in the current directory will be
listed, one per line in the output. This will include both files and
subdirectories. Each entry will include a path relative to the current
directory: for files \emph{in} the current directory, this will be |./|. The
order of results will be determined by the underlying operating system process:
unless requested \emph{via} an option, no sorting takes place.

As standard, the \meta{args} are treated as a file/path name potentially
including |?| and |*| as wildcards, for example |*.png| or |file?.txt|.
\begin{verbatim}
  l3sys-query ls '*.png'
\end{verbatim}
Some care is needed in preventing expansion of such wildcards by the shell or
\texttt{texlua} process: these are detailed in Section~\ref{sec:wildcard}. In
this section, |'| is used to indicate a character being used to suppress
expansion: this is for example normal on macOS and Linux.

Removal of entries from the listing can be achieved using the |--exclude|
option, which should be given with a \meta{xarg}, for example
\begin{verbatim}
  l3sys-query ls --exclude '*.bak' 'graphics/*'
\end{verbatim}
Directory entries starting |.| are traditionally hidden on Linux and macOS
systems: these \enquote{dot} entries are excluded unless the |--all| option is
set. Note that this will skip hidden directories entirely when used with the
|--recursive| option.

For more complex matching, the \meta{args} can be treated as a Lua pattern using
the |--pattern| (|-p|) option; this also applies to the \meta{xarg} argument to
the |--exclude| option. For example, the equivalent to wildcard |*.png| could be
obtained using
\begin{verbatim}
  l3sys-query ls --pattern '^.*%.png$'
\end{verbatim}

Since \texttt{l3sys-query} is intended primarily for use with restricted shell
escape calls from \TeX{} processes, handling of spaces is unusual. It is not
possible to quote spaces in such a call, so for example whilst
\begin{verbatim}
  l3sys-query ls "'foo *'"
\end{verbatim}
does work from the command prompt to find all files with names starting
\verb*|foo |, it would not work \emph{via} restricted shell escape. To
circumvent this, \texttt{l3sys-query} will collect all command line arguments
after any \meta{options}, and combine these as a space-separated \meta{args},
for example allowing
\begin{verbatim}
  l3sys-query ls 'foo *'
\end{verbatim}
to work. Note that the \emph{entire} \meta{args} should be surrounded by |'|
characters to prevent expansion of the~|*|.

The results returned by |ls| can be sorted using the |--sort| option. This can
be set to |none| (use the order from the file system: the default), |name| (sort
by file name) or |date| (sort by date last modified). The sorting order can be
reversed using |--reverse-sort|. Sorting normally takes account of case: this
can be suppressed with the |--ignore-case| option.

The listing can be filtered based on the type of entry using the |--type|
option. This takes a string argument, one of |d| (directory) or |f| (file).

As standard, only the path specified as part of the \meta{args} is queried.
However, if the |--recursive| (|-r|) option is set, the query is applied within
all subdirectories.

For security reasons, only paths within the current working directory can be
queried, this for example |graphics/*.png| will list all |png| files in the
|graphics| subdirectory, but |../graphics/*.png| will yield no output.

\subsection{\texttt{pwd}}

Returns the present working directory from which \texttt{l3sys-query} is run.
From within a \TeX{} run, this will (usually) be the directory containing the
main file, assuming a command such as
\begin{verbatim}
  pdflatex main.tex
\end{verbatim}
The \texttt{pwd} command is unaffected by any options.

\section{Wildcard expansion handling\label{sec:wildcard}}

The handling of wildcards needs some further comment for those using
\texttt{l3sys-query} from the command line: the \pkg{expl3} interface described
in Section~\ref{sec:expl3} handles this aspect automatically for the user.

\subsection{Shell use}

On macOS and Linux, the shell normally expands globs, which include the
wildcards |*| and |?|, before passing arguments to the appropriate command. This
can be suppressed by surrounding the argument with |'| characters, hence the
formulation
\begin{verbatim}
  l3sys-query ls '*.png'
\end{verbatim}
earlier.

On Windows, the shell does no expansion, and thus arguments are passed as-is to
the relevant command. However, in the \TeX{}~Live implementation of
\texttt{texlua}, the script interpreter expands |*| and |?| \emph{before} they
are made available to the script itself. This can only be suppressed by
including one or more characters that prevent a match: this causes the
\emph{entire} argument to be passed to the script, including whatever
\enquote{extra} characters are used. Importantly, |'| has no special meaning
here.

To allow quoting of wildcards from the shell in a platform-neutral manner,
\texttt{l3sys-query} will strip exactly one set of |'| characters around
each argument before further processing.

\subsection{Calls \emph{via} restricted shell escape}

Restricted shell escape prevents shell expansion of wildcards entirely, but
also strips out all |'| characters from arguments. Thus on macOS and
Linux, a \TeX{} call such as
\begin{verbatim}
  \input|"l3sys-query ls '*.png'"
\end{verbatim}
will work as it is converted to one without the |'| characters (the surrounding
|"| are needed to allow space-separated arguments to be passed as a whole to
|\input|).

On Windows, the rules of restricted shell escape mean that |'| cannot be
used here to prevent expansion of |*| and |?|. Rather, the character |:|
should be used: this is reserved on Windows in any case. Thus one would use
\begin{verbatim}
  \input|"l3sys-query ls :*.png:"
\end{verbatim}
on Windows; \texttt{l3sys-query} and strips out exactly one set of surrounding
|:| characters.

\section{The \LaTeX{} interface\label{sec:expl3}}

Using \texttt{l3sys-query} is not tied to access \emph{via} \pkg{expl3}, but
this is the preferred approach for the \LaTeX{} Team. Details of how to use
\texttt{l3sys-query} as an \pkg{expl3} programmer will covered in
\texttt{interface3.pdf} once the macro code is finalised.

\end{documentation}

\PrintIndex

\end{document}
