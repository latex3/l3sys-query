\iffalse meta-comment

File l3sys-query.tex Copyright (C) 2024 The LaTeX Project

-----------------------------------------------------------------------

The development version of the bundle can be found at

   https://github.com/latex3/lsys-query

for those people who are interested.

-----------------------------------------------------------------------

\fi

\documentclass{l3doc}

\begin{document}

\title{The \pkg{l3sys-query} script: System queries for LaTeX using Lua}

\author{%
  The \LaTeX{} Project\thanks{%
    Email: \href{mailto:latex-team@latex-project.org}
      {latex-team@latex-project.org}%
  }%
}

\date{Released 2024-03-03}

\maketitle
\tableofcontents

\begin{documentation}

\section{Introduction}

\TeX{} engines provide only very limited access to information about the system
they are used on: using primitives, one can for example get the size of a single
file, but not a list of files in a given location. For most documents, this is
not an issue as they are self-contained. However, for cases where
\enquote{dynamic} construction of parts of a document is needed based on file
lists or other system-dependent data, methods to obtain this from (restricted)
shell escape are desirable.

Security considerations mean that directly querying the system shell is
problematic for general use. Instead, \emph{restricted} shell escape may be used
to get many details, provided a suitable tool is available to provide the
information in a platform-neutral and security-conscious way. The Java program
\texttt{texosquery}, written by Nicola Talbot, has been available for a number
of years to provide this facility. As well as file system insight,
\texttt{texosquery} also provides for example locale data and other system
information. However, the requirement for Java means that the script is not
automatically usable when a \TeX{} system is installed.

The \LaTeX{} team have therefore provided a Lua-based script,
\texttt{l3sys-query}, which conforms to the security requirements of \TeX{} Live
using Lua to obtain the system information. This means that it can be used
\enquote{out of the box} across platforms. The facilities provided by
\texttt{l3sys-query} are more limited than \texttt{texosquery}, partly as some
information is available in modern \TeX{} systems using primitives, and partly
as the aim of \texttt{l3sys-query} is to provide information where there are
defined use cases. Requests for additional data interfaces are welcome.

\section{The command line interface}

The command line interface to 
\begin{center}
  \ttfamily
  l3sys-query \meta{cmd} [\meta{option(s)}] [\meta{spec}]
\end{center}
where \texttt{\meta{cmd}} can be one of the following:
\begin{itemize}[noitemsep]\ttfamily
  \item ls
  \item ls \meta{spec}
  \item pwd
\end{itemize}
The \meta{cmd} are described below. The result of the \meta{cmd} will be
printed to the terminal in an interactive run; in normal usage, this will be
piped to the calling \TeX{} process.

As well as these targets, the script recognises the options
\begin{itemize}
\end{itemize}
The action of these options on the appropriate \meta{cmd(s)} is detailed below.

\subsection{\texttt{pwd}}

Returns the present working directory from which \texttt{l3sys-query} is run.
From within a \TeX{} run, this will (usually) be the directory containing the
main file, assuming a command such as
\begin{verbatim}
  pdflatex main.tex
\end{verbatim}
The \texttt{pwd} command is unaffected by any options.

\section{The \LaTeX{} interface}

Using \texttt{l3sys-query} is not tied to access \emph{via} \pkg{expl3}, but
this is the preferred approach for the \LaTeX{} Team. Details of how to use
\texttt{l3sys-query} as an \pkg{expl3} programmer are covered in
\texttt{interface3.pdf}.

\end{documentation}

\PrintIndex

\end{document}
