\iffalse meta-comment

File l3sys-query.tex Copyright (C) 2024 The LaTeX Project

-----------------------------------------------------------------------

The development version of the bundle can be found at

   https://github.com/latex3/l3sys-query

for those people who are interested.

-----------------------------------------------------------------------

\fi

\documentclass{l3doc}

\begin{document}

\title{The \pkg{l3sys-query} script: System queries for LaTeX using Lua}

\author{%
  The \LaTeX{} Project\thanks{%
    Email: \href{mailto:latex-team@latex-project.org}
      {latex-team@latex-project.org}%
  }%
}

\date{Released 2024-03-03}

\maketitle
\tableofcontents

\begin{documentation}

\section{Introduction}

\TeX{} engines provide only very limited access to information about the system
they are used on: using primitives, one can for example get the size of a single
file, but not a list of files in a given location. For most documents, this is
not an issue as they are self-contained. However, for cases where
\enquote{dynamic} construction of parts of a document is needed based on file
lists or other system-dependent data, methods to obtain this from (restricted)
shell escape are desirable.

Security considerations mean that directly querying the system shell is
problematic for general use. Instead, \emph{restricted} shell escape may be used
to get many details, provided a suitable tool is available to provide the
information in a platform-neutral and security-conscious way. The Java program
\texttt{texosquery}, written by Nicola Talbot, has been available for a number
of years to provide this facility. As well as file system insight,
\texttt{texosquery} also provides for example locale data and other system
information. However, the requirement for Java means that the script is not
automatically usable when a \TeX{} system is installed.

The \LaTeX{} team have therefore provided a Lua-based script,
\texttt{l3sys-query}, which conforms to the security requirements of \TeX{} Live
using Lua to obtain the system information. This means that it can be used
\enquote{out of the box} across platforms. The facilities provided by
\texttt{l3sys-query} are more limited than \texttt{texosquery}, partly as some
information is available in modern \TeX{} systems using primitives, and partly
as the aim of \texttt{l3sys-query} is to provide information where there are
defined use cases. Requests for additional data interfaces are welcome.

\section{The command line interface}

The command line interface to 
\begin{center}
  \ttfamily
  l3sys-query \meta{cmd} [\meta{option(s)}] [\meta{spec}]
\end{center}
where \texttt{\meta{cmd}} can be one of the following:
\begin{itemize}[noitemsep]\ttfamily
  \item ls
  \item ls \meta{spec}
  \item pwd
\end{itemize}
The \meta{cmd} are described below. The result of the \meta{cmd} will be
printed to the terminal in an interactive run; in normal usage, this will be
piped to the calling \TeX{} process. Results containing path separators
\emph{always} use~|/|, irrespective of the platform in use.

As well as these targets, the script recognizes the options
\begin{itemize}
  \item |--exclude| Specification for directory entries to exclude
  \item |--exclude-dot| Exclude \enquote{dot} entries from directory listings
  \item |--ignore-case| Ignores case when sorting directory listings
  \item |--recursive| (|-r|) Enables recursive searching during directory
    listings
  \item |--reverse-sort| Causes sorting to go from highest to lowest rather
    than lowest to highest
  \item |--sort| Sets the method used to sort entries returned by |ls|
  \item |--type| Selects the type of entry returned by |ls|
\end{itemize}
The action of these options on the appropriate \meta{cmd(s)} is detailed below.

\subsection{\texttt{ls [\meta{spec}]}}

Lists the contents of one or more directories, in a manner somewhat reminiscent
of the Unix command |ls| or the Windows command |dir|. The exact nature of the
output will depend on the \meta{spec}, is given, along with the prevailing
options.

When no \meta{spec} is given, all entries in the current directory will be
listed, one per line in the output. This will include both files and
subdirectories. Each entry will include a path relative to the current
directory: for files \emph{in} the current directory, this will be |./|. The
order of results will be determined by the underlying operating system process:
unless requested \emph{via} an option, no sorting takes place.

As standard, the \meta{spec} is treated as a file glob, for example |*.png|. To
prevent wildcards in such globs from being expanded at the wrong time, they
should always be surround by |'| characters,\footnote{On Windows, the shell does
not expand globs, but \texttt{texlua} does: using \texttt{'} is a
platform-neutral way to avoid any expansion.} for example
\begin{verbatim}
  l3sys-query ls '*.png'
\end{verbatim}
Removal of entries from the listing can be achieved using the |--exclude| option,
which should be given with a \meta{spec}, for example
\begin{verbatim}
  l3sys-query ls --exclude '*.bak' 'graphics/*'
\end{verbatim}
Removal of entr
Directory entries starting |.| are traditionally hidden on Linux and macOS systems,
and can be excluded from listings using the |--exclude-dot| option. Note that
this will skip hidden directories entirely when used with the |--recursive|
option.

Since \texttt{l3sys-query} is intended primarily for use with restricted shell
escape calls from \TeX{} processes, handling of spaces is unusual. It is not
possible to quote spaces in such a call, so for example whilst
\begin{verbatim}
  l3sys-query ls "'foo *'"
\end{verbatim}
does work from the command prompt to find all files with names starting
\verb*|foo |, it would not work \emph{via} restricted shell escape. To
circumvent this, \texttt{l3sys-query} will collect all command line arguments
after any \meta{options}, and combine these as a space-separated \meta{spec},
for example allowing
\begin{verbatim}
  l3sys-query ls 'foo *'
\end{verbatim}
to work. Note that the \emph{entire} \meta{spec} should be surrounded by |'|
characters to prevent expansion of the~|*|.

The results returned by |ls| can be sorted using the |--sort| option. This
can be set to |none| (use the order from the file system: the default),
|name| (sort by file name or |date| (sort by date last modified). The sorting
order can be reversed using |--reverse-sort|. Sorting normally takes account
of case: this can be suppressed with the |--ignore-case| option.

The listing can be filtered based on the type of entry using the |--type|
option. This takes a string argument, one of |d| (directory) or |f| (file).

As standard, only the path specified as part of the \meta{spec} is queried.
However, if the |--recursive| (|-r|) option is set, the query is applied within
all subdirectories.

For security reasons, only paths within the current working directory can be
queried, this for example |graphics/*.png| will list all |png| files in the
|graphics| subdirectory, but |../graphics/*.png| will yield no output.

\subsection{\texttt{pwd}}

Returns the present working directory from which \texttt{l3sys-query} is run.
From within a \TeX{} run, this will (usually) be the directory containing the
main file, assuming a command such as
\begin{verbatim}
  pdflatex main.tex
\end{verbatim}
The \texttt{pwd} command is unaffected by any options.

\section{The \LaTeX{} interface}

Using \texttt{l3sys-query} is not tied to access \emph{via} \pkg{expl3}, but
this is the preferred approach for the \LaTeX{} Team. Details of how to use
\texttt{l3sys-query} as an \pkg{expl3} programmer are covered in
\texttt{interface3.pdf}.

\end{documentation}

\PrintIndex

\end{document}
